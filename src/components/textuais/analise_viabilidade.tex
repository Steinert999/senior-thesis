\newcommand{\footnotegantt}{
    \footnote{O Gráfico de Gantt, também conhecido como Diagrama de Gantt,
        é uma ferramenta visual para controlar o cronograma de um projeto ou
        de uma programação de produção, ajudando a avaliar os prazos de
        entrega e os recursos críticos.
        Disponível em:
        \url{https://www.nomus.com.br/blog-industrial/grafico-de-gantt/}
    }
}
\chapter{Análise de Viabilidade da Pesquisa}
\label{ch:analise-de-viabilidade-da-pesquisa}

Quanto à viabilidade desse projeto, ele possui qualidades que fundamentam e
garantem sua viabilidade.
No campo da contribuição científica, ela é considerada factível, pois se esforça para
apoiar e contribuir com a área de pesquisa relacionada à autenticação de documentos
digitais, particularmente aqueles relativos aos~\acrlongpl{tcc}.

Em termos de contribuições sociais, ele se alinha devido à eficácia desse trabalho,
em que, por exemplo, servirá como um auxílio substancial na validação de tarefas
realizadas no campo acadêmico da ciência da computação, simplificando assim o
processo para educadores e alunos envolvidos.

Na autenticação de~\acrshortpl{pdf}, vimos que há uma lacuna muito grande,
devido ao uso majoritário de documentos impressos.
O uso de tal, auxilia na facilidade de criar documentos fraudulentos, algo que,
há maior dificuldade em documentos digitais.

A inclusão da contribuição tecnológica, por outro lado, está englobada nos algoritmos
de autenticação, incorporando-os ao campo da legalidade, mais precisamente dentro
da estrutura constitucional estabelecida pelo~\citeauthor*{govbr2020}.

Em questão de execução computacional, o trabalho se afirma possivel,
tendo em vista a diversidade de tecnologias possiveis para criação de
~\acrshortpl{api}.
Para este trabalho, as tecnologias utilizadas são comumente utilizadas para
desenvolvimento de~\acrshortpl{api}, no qual garantem facilidade, organização
e desempenho para o mesmo.
Também conforme dito em capítulos anteriores, a autenticidade dos documentos
e legalidade dos mesmos, são garantidas, pelo ~\citeauthor*{govbr2020},
visto que o mesmo proporciona leis, e inclusive uma~\acrshort{api} que
autentica e faz o intermédio deste trabalho.

Relacionando o projeto a custos do mesmo, esta pesquisa usa em sua totalidade
projetos \textit{open-source}, e projetos que sejam gratuitos (inclusive o
disposto pelo~\citeauthor*{govbr2020}), assim tornando-o mais adequado
para sua continuidade sem custos.
\newpage
\section{Cronograma de atividades para desenvolvimento da pesquisa}
\label{sec:cronograma-de-desenvolvimento-de-pesquisa}

A tabela a seguir, definida como um diagrama Gantt\footnotegantt, representa o
cronograma do desenvolvimento desta pesquisa:


\ganttset{
    group/.append style={draw=black, fill=green!50, drop shadow },
    milestone/.append style={draw=black, fill=red!50},
    bar/.append style={fill=blue!20, draw=black, rounded corners=2mm,
    drop shadow},
    bar inline label node/.append style={left=1pt}
}

\begin{table}[h!]
    \begin{ganttchart}
    [
        hgrid, vgrid,
        x unit=1cm,
        y unit chart=.5cm,
        time slot format=isodate,
        time slot unit=month
    ]{2023-06-01}{2023-11-20}
        \gantttitlecalendar{month=shortname}\\
        
        \ganttgroup{Desenv. do documento}{2023-06-01}{2023-11-20}\\
        \ganttbar{Definição do tema}{2023-06-01}{2023-07-30}\\
        \ganttlinkedbar{Procura de artigos}{2023-08-01}{2023-09-30}\\
        \ganttbar{Escrita}{2023-06-01}{2023-11-20}\\
        
        \ganttlinkedmilestone{Finalização do documento}{2023-11-20}\\
        
        \ganttgroup{Desenvolvimento da
        \acrshort{api}}{2023-07-01}{2023-11-20}\\
        
        
        \ganttbar{Arquitetura da \acrshort{api}}
        {2023-07-01}{2023-08-30} \\
        \ganttlinkedbar{Regras de negócio}{2023-07-15}{2023-09-30} \\
        \ganttlinkedbar{Escrita do código da \acrshort{api}}
        {2023-09-30}{2023-10-30} \\
        \ganttlinkedbar{Testes}{2023-10-30}{2023-11-20} \\
        
        \ganttlinkedmilestone{Finalização geral}{2023-11-20} \\
    
    \end{ganttchart}
    \sourcesearchfootnote
    \label{tab:cronograma-pesquisa}
    \caption[Cronograma de desenvolvimento]{Cronograma de
    atividades para o desenvolvimento da pesquisa}
\end{table}