\section{Problematização do tema}
As informações estão facilmente disponíveis em formato digital graças ao uso generalizado da Internet e ao rápido crescimento das tecnologias da informação. Como as impressoras estão prontamente disponíveis e são convenientes, o conteúdo digital pode ser impresso livremente em documentos. Por outro lado, os documentos impressos podem ser alterados ilegalmente por vários motivos criminosos, incluindo documentos falsos, dinheiro falso, violações de direitos autorais, etc.\cite{tsai2019}

Mesmo que os documentos digitais estejam amplamente dispersos, os documentos impressos ainda são amplamente aceitos e simples de distribuir porque as impressoras se tornaram itens comuns. Entretanto, os papéis digitais impressos produzidos por esses dispositivos têm o potencial de fornecer informações sobre crimes como falsificação, fabricação de documentos, fraude em contas ou loterias, entre outros. É necessário identificar a origem da impressora, pois a violação de direitos autorais e a falsificação criminosa continuam sendo estudadas para novas mídias.\cite{tsai2019}

O aumento do uso de documentos digitais é resultado da implementação do sistema de informações no setor público. Por serem aceitos pela lei, os documentos digitais substituem os impressos. O documento de permissão é um exemplo de um documento impresso que ainda é importante porque ainda é exigido como prova admissível. Conforme exigido por regulamentos administrativos, os documentos de permissão são cruciais. Devido à importância dos documentos de permissão, houve vários casos de falsificação, por exemplo. Portanto, é necessário garantir a legitimidade desses documentos.\cite{arief2019}

A capacidade de criar documentos impressos falsos melhorou drasticamente devido aos avanços na tecnologia de impressão e digitalização. A digitalização e reimpressão de documentos legais impressos é uma técnica de falsificação fácil para os falsificadores. Naturalmente, os fraudadores podem alterar os documentos digitalizados usando técnicas de processamento de imagens para fazer com que os documentos recém-impressos se pareçam com os autênticos. Várias técnicas anti-falsificação, incluindo hologramas, marcas d'água e materiais impressos específicos (como tinta), foram sugeridas para detectar documentos genuínos dos falsificados. Entretanto, essas técnicas exigem determinados equipamentos para gerar documentos legais e realizar a autenticação, o que aumenta seu custo.\cite{zhang2019}

Diante destas situações, este trabalho vem por auxiliar na resolução dessas problemáticas através da implementação de uma \acrfull{api}, que, mencionado com o \acrlong{ies}, deverá autenticar e incluir as assinaturas digitais dos relacionados ao \acrfull{tcc} depositado no mesmo, com as devidas especificidades definidas pela \acrshort{ies}.