\chapter{Introdução}\label{ch:introducao}

As informações estão facilmente disponíveis em formato digital graças ao uso
generalizado da Internet e ao rápido crescimento das tecnologias da informação.
Como as impressoras estão prontamente disponíveis e são convenientes, o conteúdo
digital pode ser impresso livremente em documentos.
Por outro lado, os documentos impressos podem ser alterados ilegalmente por
vários motivos criminosos, incluindo documentos falsos, dinheiro falso,
violações de direitos autorais, entre outros\cite{tsai2019}.

O aumento do uso de documentos digitais é resultado da implementação do
sistema de informações no setor público.
Por serem aceitos pela lei, os documentos digitais substituem os impressos.
Devido à importância dos documentos de permissão, houve vários casos de
falsificação, por exemplo.
Portanto, é necessário garantir a legitimidade desses documentos\cite{
    arief2019}.

Neste trabalho, será apresentado uma solução para mitigar o uso de documentos
impressos, e facilitar a autenticação digital, desenvolvendo uma~\acrfull{api},
com foco em assinar e autenticar digitalmente os~\acrfullpl{tcc} enviados
referentes ao curso de ciência da computação da instituição~\acrlong{iesgf}.
\section{Tema}\label{sec:tema}
Autenticação em Sistemas de Informação
\section{Delimitação do tema}\label{sec:delimitacao-do-tema}

Este projeto se dará durante o período de agosto de 2023 até agosto
de 2024, procura criar uma implementação focada em autenticar um ou
mais~\acrfull{tcc}~fornecidos ao sistema pelos alunos do~\acrfull{iesgf}.
A autenticação consiste em passar por uma série de etapas,
onde, após a inclusão do trabalho, ficará ao encargo do orientador
responsável.
Em seguida após a aprovação do orientador, será encaminhado o trabalho
para a banca examinadora, semelhante a finalizar a autenticação.
O foco deste trabalho se correlaciona à~\acrshort{api} que fornece o
fluxo de autenticação, conforme Figura~\ref{fig:diagrama-sequencia-api}.

No desenvolvimento será aplicado as principais técnicas de
programação, para trazer melhor legibilidade e também desempenho
para a implementação da~\acrshort{api}.
Já as tecnologias envolvidas neste trabalho, foram utilizados o \acrfull{os}
Windows, também como principal linguagem o~\acrfull{kotlin} na
versão 1.9.0, utilizando o ecossistema do~\textit{framework} Spring
atualmente na versão 3.2.x, tais como, Spring Web, Spring Data,
Spring Security, entre outros.
Também foi utilizado a plataforma~\textit{cloud} Google Cloud, para
gerenciar toda a infraestrutura de~\textit{deploy} da~\acrshort{api},
assim como o banco de dados relacional utilizado, o PostgreSQL,
atualmente na versão 16.0.
Para complementar e auxiliar a autenticação dos arquivos PDF, o
~\citeauthor*{decreto112020} disponibiliza uma~\acrshort{api} de
autenticação de documentos seguindo a legislação descrita no
~\citetitle*{decreto112020}~\cite{decreto112020}.

\begin{figure}[h!]
    \centering
    \begin{tikzpicture}
        \begin{umlseqdiag}
            \umlobject[x=2, y=1, no ddots]{Autenticador}
            \umlactor[x=0, no ddots]{Aluno}
            \umlactor[x=4, no ddots]{Orientador}
            \umlactor[x=6, no ddots]{Banca}
            \begin{umlcall}[op=Envia Arquivo, return=Arquivo Final]
            {Aluno}{Autenticador}
                \begin{umlcall}[op=Notifica, return=Arquivo]
                {Autenticador}{ Orientador}
                    \begin{umlcall}[op=Notifica]
                    {Autenticador}{Aluno}
                    \end{umlcall}
                \end{umlcall}
                \begin{umlcall}[op=Notifica, return=Arquivo]
                {Autenticador}{Banca}
                    \begin{umlcall}[op=Notifica]
                    {Autenticador}{Aluno}
                    \end{umlcall}
                \end{umlcall}
            
            \end{umlcall}
        \end{umlseqdiag}
    \end{tikzpicture}
    \caption[Diagrama de sequência Autenticação \acrshort{api}]{
        Diagrama de sequência com o fluxo de autenticação
        da \acrshort{api}}
    \label{fig:diagrama-sequencia-api}
\end{figure}
\section{Problematização do Tema}

As informações estão facilmente disponíveis em formato digital gra
ças ao uso generalizado da Internet e ao rápido crescimento das
tecnologias da informação.
Como as impressoras estão prontamente disponíveis e são
convenientes, o conteúdo digital pode ser
impresso livremente em documentos.
Por outro lado, os documentos impressos podem ser alterados
ilegalmente por vários motivos criminosos, incluindo documentos
falsos, dinheiro falso, violações
de direitos autorais, etc\cite{tsai2019}.

Mesmo que os documentos digitais estejam amplamente dispersos, os
documentos impressos ainda são amplamente aceitos e simples de
distribuir porque as impressoras se tornaram itens comuns.
Entretanto, os papéis digitais impressos produzidos por esses
dispositivos têm o potencial de fornecer informações sobre crimes
como falsificação, fabricação de documentos, fraude em contas ou
loterias, entre outros.
É necessário identificar a origem da impressora, pois a violação de
direitos autorais e a falsificação criminosa continuam sendo
estudadas para novas mídias\cite{tsai2019}.

O aumento do uso de documentos digitais é resultado da implementa
ção do sistema de informações no setor público.
Por serem aceitos pela lei, os documentos digitais substituem os
impressos.
O documento de permissão é um exemplo de um documento impresso que
ainda é importante porque ainda é exigido como prova admissível.
Conforme exigido por regulamentos administrativos, os documentos de
permissão são cruciais.
Devido à importância dos documentos de permissão, houve vários
casos de falsificação, por exemplo.
Portanto, é necessário garantir a legitimidade desses documentos\cite{
    arief2019}.

A capacidade de criar documentos impressos falsos melhorou
drasticamente devido aos avanços na tecnologia de impressão e
digitalização.
A digitalização e reimpressão de documentos legais impressos é uma
técnica de falsificação fácil para os falsificadores.
Naturalmente, os fraudadores podem alterar os documentos
digitalizados usando técnicas de processamento de imagens para
fazer com que os documentos recém-impressos se pareçam
com os autênticos.
Várias técnicas anti-falsificação, incluindo hologramas, marcas
d'água e materiais impressos específicos (como tinta), foram
sugeridas para detectar documentos genuínos dos falsificados.
Entretanto, essas técnicas exigem determinados equipamentos para
gerar documentos legais e realizar a autentica ção, o que aumenta
seu custo\cite{zhang2019}.

Considerando a facilidade de falsificação de documentos impressos e digitais,
este estudo visa auxiliar por meio da implementação de uma~\acrshort{api} que,
em conformidade com o~\acrlong{iesgf}, irá autenticar e incluir as assinaturas
digitais dos envolvidos nos \acrfullpl{tcc} depositados, seguindo as normas
estabelecidas pela instituição.
\section{Hipóteses}

\begin{enumerate}[label=\alph*)]
    \item Com o desenvolvimento desta \acrshort{api}, será possível autenticar devidamente os trabalhos enviados?
    \item Os documentos autenticados pela \acrshort{api} poderão ser usados legalmente autenticados?
    \item Os documentos autenticados pela \acrshort{api} estarão conforme o padrão aceito pela instituição \acrfull{iesgf}?
\end{enumerate}
\section{Objetivos}

Esta seção trata-se da definição dos objetivos abordados para execução deste trabalho.

\subsection{Objetivo geral}

Implementar a \acrshort{api}, utilizando métodos de autenticação certificados e permitidos para uso na universidade.

\subsection{Objetivos específicos}
\underline{Os objetivos específicos da pesquisa são:}

\newcommand{\buscaReferencia}{
 Buscar conteúdo referencial para o artigo, contendo conceitos de autenticação e assinatura digital.
}

\newcommand{\arquitetura}{
 Estruturar a arquitetura do sistema, para atender os requisitos necessários e especificos da universidade.
}

\newcommand{\implementacao}{
  Desenvolver a \acrshort{api}, conforme as especificações acordadas no objetivo anterior
}

\newcommand{\testes}{
    Testar a \acrshort{api}, juntamente com com um usuário piloto, afim de confirmar a funcionalidade, e garantir que não haja nenhuma divergencia na especificação.
}

\begin{enumerate}[label=\alph*)]
   \item  \buscaReferencia
   \item  \arquitetura
   \item  \implementacao
   \item  \testes
\end{enumerate}