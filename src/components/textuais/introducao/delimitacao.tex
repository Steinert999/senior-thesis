\section{Delimitação do tema}\label{sec:delimitacao-do-tema}

Este projeto se dará durante o período de agosto de 2023 até agosto
de 2024, procura criar uma implementação focada em autenticar um ou
mais \acrfull{tcc} fornecidos ao sistema pelos alunos do
\acrfull{iesgf}.
A autenticação consiste em passar por uma série de etapas,
onde, após a inclusão do trabalho, ficará ao encargo do orientador
responsável.
Em seguida após a aprovação do orientador, será encaminhado o trabalho
para a banca examinadora, semelhante a finalizar a autenticação.
O foco deste trabalho se correlaciona à \acrshort{api} que fornece o
fluxo de autenticação, conforme~\ref{fig:fluxoapi}.

No desenvolvimento será aplicado as principais técnicas de
programação, para trazer melhor legibilidade e também desempenho
para a implementação da \acrshort{api}.
Já as tecnologias envolvidas neste trabalho, foram utilizados o \acrfull{os}
Windows, também como principal linguagem o \acrfull{kotlin} na
versão 1.9.0, utilizando o ecossistema do \textit{framework} Spring
atualmente na versão 3.2.x, tais como, Spring Web, Spring Data,
Spring Security, entre outros.
Também foi utilizado a plataforma \textit{cloud} Google Cloud, para
gerenciar toda a infraestrutura de \textit{deploy} da \acrshort{api},
assim como o banco de dados relacional utilizado, o PostgreSQL,
atualmente na versão 16.0.
Para complementar e auxiliar a autenticação dos arquivos PDF, o
Governo Federal do Brasil disponibiliza uma \acrshort{api} de
autenticação de documentos, \citetitle{decreto112020}

\begin{figure}[h!]
    \centering
    \begin{tikzpicture}
        \begin{umlseqdiag}
            \umlobject[x=2, y=1, no ddots]{Autenticador}
            \umlactor[x=0, no ddots]{Aluno}
            \umlactor[x=4, no ddots]{Orientador}
            \umlactor[x=6, no ddots]{Banca}
            \begin{umlcall}[op=Envia Arquivo, return=Arquivo Final]
            {Aluno}{Autenticador}
                \begin{umlcall}[op=Notifica, return=Arquivo]
                {Autenticador}{ Orientador}
                    \begin{umlcall}[op=Notifica]
                    {Autenticador}{Aluno}
                    \end{umlcall}
                \end{umlcall}
                \begin{umlcall}[op=Notifica, return=Arquivo]
                {Autenticador}{Banca}
                    \begin{umlcall}[op=Notifica]
                    {Autenticador}{Aluno}
                    \end{umlcall}
                \end{umlcall}
            
            \end{umlcall}
        \end{umlseqdiag}
    \end{tikzpicture}
    \caption[Diagrama de sequência Autenticação \acrshort{api}]{
        Diagrama de sequência com o fluxo de autenticação
        da \acrshort{api}}
    \label{fig:diagrama-sequencia-api}
\end{figure}