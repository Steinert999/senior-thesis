\section{Delimitação do tema}\label{sec:delimitacao-do-tema}

Este projeto se dará durante o período de agosto de 2023 até agosto
de 2024, procura criar uma implementação focada em autenticar um ou
mais \acrfull{tcc} fornecidos ao sistema pelos alunos do
\acrfull{iesgf}.
A autenticação consiste em passar por uma série de etapas,
onde, após a inclusão do trabalho, ficará ao encargo do orientador
responsável.
Em seguida após a aprovação do orientador, será encaminhado o trabalho
para a banca examinadora, semelhante a finalizar a autenticação.
O foco deste trabalho se correlaciona à \acrshort{api} que fornece o
fluxo de autenticação, conforme~\ref{fig:fluxoapi}.
\begin{figure}[h!]
    \centering
    \begin{sequencediagram}
        \newthread{AU}{Autenticador}{}
        \newthread{A}{Aluno}{}
        \newinst{O}{Orientador}{}
        \newinst{B}{Banca examinadora}{}
        \begin{call}{A}{Solicitação}{AU}{Arquivo completo}
            \begin{call}{AU}{Notifica}{O}{Autenticado}
                \postlevel
            \end{call}
            \begin{call}{AU}{Notifica}{B}{Autenticado}
                \postlevel
            \end{call}
        \end{call}
    \end{sequencediagram}
    \caption[Diagrama de sequência Autenticação \acrshort{api}]{
        \textit{Diagrama de sequência com o fluxo de autenticação
        da \acrshort{api}}}
    \label{fig:fluxoapi}
\end{figure}
No desenvolvimento será aplicado as principais técnicas de programa
ção orientada a objetos e programação funcional, para trazer
melhor legibilidade e também desempenho para a implementação da \acrshort{api}.
Para o desenvolvimento da \acrshort{api}, será utilizado a \acrfull{ide} Intellij Idea.
A versão do \acrshort{kotlin} será 1.9.0, versão mais atual do
mesmo no presente momento.
De forma a validar a legalidade da autenticação dos \acrlong{tcc},
será utilizado a \acrshort{api} fornecida pelo governo para
autenticação de documentos, seguindo as diretrizes do decreto
nº 10.543 de 13 de novembro de 2020~\cite{decreto112020}