\section{Delimitação do tema}\label{sec:delimitacao-do-tema}

Este projeto será realizado no período de agosto de 2023 a agosto de 2024.
Seu objetivo principal é desenvolver uma implementação voltada para a
autenticação de \acrfullpl{tcc} fornecidos ao sistema pelos alunos do
\acrfull{iesgf}.
A autenticação é um procedimento constituído por múltiplas etapas.
Primeiramente, o aluno realiza o envio do arquivo.
Após o envio, o sistema emite uma notificação ao orientador, solicitando a
validação e assinatura do documento.
Essa ação confirma a necessidade de autenticação por parte do sistema.
Posteriormente, ocorre o encaminhamento do trabalho à banca examinadora,
seguindo um fluxo semelhante ao realizado pelo orientador, depois de obter
aprovação deste último.
A representação visual do procedimento do sistema encontra-se exemplificada
na figura~\ref{fig:diagrama-sequencia-api}.

\begin{figure}[h!]
    \centering
    \begin{tikzpicture}
        \begin{umlseqdiag}
            \umlobject[x=2, y=1, no ddots]{Autenticador}
            \umlactor[x=0, no ddots]{Aluno}
            \umlactor[x=4, no ddots]{Orientador}
            \umlactor[x=6, no ddots]{Banca}
            \begin{umlcall}[op=Envia Arquivo, return=Arquivo Final]
            {Aluno}{Autenticador}
                \begin{umlcall}[op=Notifica, return=Arquivo]
                {Autenticador}{ Orientador}
                    \begin{umlcall}[op=Notifica]
                    {Autenticador}{Aluno}
                    \end{umlcall}
                \end{umlcall}
                \begin{umlcall}[op=Notifica, return=Arquivo]
                {Autenticador}{Banca}
                    \begin{umlcall}[op=Notifica]
                    {Autenticador}{Aluno}
                    \end{umlcall}
                \end{umlcall}
            
            \end{umlcall}
        \end{umlseqdiag}
    \end{tikzpicture}
    \caption[Diagrama de sequência Autenticação \acrshort{api}]{
        Diagrama de sequência com o fluxo de autenticação
        da \acrshort{api}}
    \sourcesearchfootnote
    \label{fig:diagrama-sequencia-api}
\end{figure}
\newpage

No desenvolvimento, serão aplicadas as principais técnicas de programação para
trazer melhor legibilidade e também desempenho para a implementação da
~\acrfull{api}.
Quanto às tecnologias envolvidas neste trabalho, será utilizado o~\acrfull{os}
Windows, bem como a linguagem principal~\acrfull{kotlin}, na versão 1.9.0,
acompanhada pelo ecossistema do~\textit{framework} Spring, atualmente na versão
3.2.x, tais como Spring Web, Spring Data, Spring Security, entre outros.
Além disso, será utilizada a plataforma de nuvem Google Cloud para gerenciar
toda a infraestrutura de~\textit{deployment} da~\acrshort{api}, assim como o
banco de dados relacional PostgreSQL, atualmente na versão 16.0. De modo a
complementar este trabalho, especificamente na autenticação dos arquivos,
o~\citeauthor*{govbr2020} disponibiliza uma~\acrshort{api} para autenticação
generalizada de documentos, conforme a legislação estabelecida no
~\citetitle*{govbr2020}~\cite{govbr2020}.
