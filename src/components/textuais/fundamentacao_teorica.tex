\chapter{Fundamentação Teórica}
\section{Autenticação}
\subsection{Tipos de Autenticação}
\subsubsection{Autenticação de dois fatores}
\subsubsection{Multifator}
\subsection{Assinatura Digital}
\subsection{Normas para autenticação}

%\section{\acrlong{kotlin}}
%\acrshort{kotlin} é uma linguagem de programação tipada estaticamente que, em comparação com \acrfull{java}, diminui a verbosidade da linguagem (por exemplo, os ponto e vírgula são opcionais como terminadores de declaração ';') \cite{goismateus2019}. 
%
%As aplicações escritas em Kotlin utilizam também a \acrfull{jvm} como base.  O código escrito em Kotlin é traduzido em código binário Java pelo \acrfull{kotlinc} e depois executado pela \acrshort{jvm}. \acrshort{kotlin} é compatível com \acrshort{java} e outras linguagens \acrshort{jvm} devido a isto (por exemplo, \acrshort{scala} ou \acrshort{groovy}).
%
%Duas perspectivas sobre esta interoperabilidade são possíveis; primeiro, os criadores do \acrshort{kotlin} podem utilizar bibliotecas (tais como \acrfull{jar}) criadas em outras linguagens que executam em cima da \acrshort{jvm}, tais como \acrshort{java}; Em segundo lugar, os programadores podem combinar o \acrshort{kotlin} com outras linguagens \acrshort{jvm} para construir aplicações.
%
%\subsection{Funcionalidades do \acrshort{kotlin}}
%Linguagens modernas de programação como \acrshort{java} e \acrshort{kotlin} rodam em cima da \acrshort{jvm}. 
%
%Entretanto, o \acrshort{kotlin} difere do \acrshort{java} de algumas formas fundamentais. Uma das diferenças mais significativas é que, além do paradigma orientado ao objeto, \acrshort{kotlin} introduz a programação funcional, que permite o uso de funções de alta ordem, por exemplo.
%Segundo a documentação oficial, a seguinte lista apresenta algumas características de Kotlin não presentes em Java: 
%\begin{itemize}
%    \item A combinação de Expressões \textit{Lambda} e funções \textbf{Inline};
%    \item funções de extensão
%    \item \textit{Null-safety}
%    \item \textit{Smart casts}
%    \item \textit{First-class delegation}
%    \item Inferência de tipo para tipos de variáveis e propriedades
%    \item \textit{Range operator}
%    \item \textit{Data class}
%    \item \textit{Interfaces separadas para coleções somente leitura e mutáveis}
%    \item Co-rotinas
%\end{itemize}


