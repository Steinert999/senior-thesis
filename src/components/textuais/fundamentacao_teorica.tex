\chapter{Fundamentação Teórica}\label{ch:fundamentacao-teorica}

Neste capítulo, estão apresentados as relações de base téoricas que
enviesam este trabalho e contribuem com o entedimento do tema.
De forma geral, foi abordado assuntos relacionados à assinatura
digital, \acrfull{api}, autenticação de arquivos, tipos de
autenticação e normas para autenticação.
\section{\acrlong{api}}\label{sec:api}

Segundo~\textcite{borgogno2019} as~\acrfullpl{api},
são coleções de protocolos que especificam os meios de comunicação
entre diferentes componentes de software.
De acordo com~\textcite{biehl2015}, generalizando, as~\acrshortpl{api} não
têm uma interface, não são graficamente aparentes, ou seja, não possuem uma
interface gráfica, e não são usadas diretamente pelos usuários finais.
Em vez disso, as~\acrshortpl{api} funcionam em segundo plano e só são chamadas
diretamente por outros programas.
As~\acrshortpl{api} são utilizadas na integração de dois ou mais sistemas de
software e na comunicação máquina a máquina.
Os desenvolvedores que usam~\acrshortpl{api} para criar aplicativos ou
soluções são os únicos que lidam diretamente com elas.
De acordo com~\textcite{russel2019} pode haver funções ou recursos que um
site queira compartilhar com outros sites ou aplicativos, assim podendo
utilizar esses protocolos, para poderem utilizar os serviços fornecidos.
A Figura~\ref{fig:diagrama-fluxo-api} exemplifica as relações e
comunicações entre diversos sistemas e outras~\acrshortpl{api}.

Para~\textcite{borgogno2019} as~\acrshortpl{api} melhoram a
interoperabilidade entre várias camadas e facilitam o compartilhamento de
fluxos ou conjuntos de dados entre os detentores de dados, simplificando o
acesso de uma organização aos dados coletados por outra.

\begin{figure}[h!]
    \centering
    \begin{tikzpicture}[
        ext/.style={circle,
        draw,
        fill=blue!5,
        thin,
        inner sep=0pt,
        align=center,
        text width=1.8cm,
        },
        api/.style={circle,
        draw,
        fill=green!5,
        thin,
        inner sep=0pt,
        align=center,
        text width=2cm,
        },
        db/.style={cylinder,
        draw,
        cylinder uses custom fill,
        cylinder body fill=yellow!20,
        cylinder end fill=yellow!40,
        aspect=0.25,
        shape border rotate=90,
        height=3cm,
        minimum height=1.5cm,
        minimum width=1.5cm
        }]
        \node[ext](web){Web Applications};
        \node[ext](mobile)[right=0.2cm of web]{Mobile Applications};
        \node[ext](desktop)[right=0.2cm of mobile]{Desktop Applications};
        \node[ext](backend)[right=0.2cm of desktop]{Backend Servers};
        \node[ext](other)[right=0.2cm of backend]{Other \acrshortpl{api}};
        \node[api](api)[below=of desktop]{\acrshort{api}}
        \node[db](db)[below=of api]{Database}
        
        % requests
        \draw[thick, ->]
        (web) edge [bend left=-45] (api)
        (mobile) edge [bend left=-45] (api)
        (desktop) edge[bend left=-10] (api)
        (backend) edge [bend right=-45] (api)
        (other) edge [bend right=-45] (api);
        %responses
        \draw[dashed, ->]
        (api) edge [bend right=-55] (web)
        (api) edge [bend right=-55] (mobile)
        (api) edge[bend right=10] (desktop)
        (api) edge[bend left=-55] (backend)
        (api) edge[bend left=-55] (other);
        
        \draw[thick, ->]
        (api) edge[bend left=-10] (db);
        \draw[dashed, ->]
        (db) edge[bend right=10] (api);
    \end{tikzpicture}
    \caption[Fluxo de uma \acrshort{api}]{
        Diagrama mostrando o fluxo de uma \acrshort{api} com diversos clientes}
    \label{fig:diagrama-fluxo-api}
\end{figure}

\newpage

Do ponto de vista tecnológico, segundo~\textcite{russel2019} os serviços
on-line não podem ser combinados ou integrados sem~\acrshortpl{api}.
Os sites e aplicativos móveis modernos e interativos são possíveis graças às
~\acrshortpl{api}.
Entretanto, quando é acessado sites e aplicativos conhecidos, as
~\acrshortpl{api} podem ser exploradas por desenvolvedores e anunciantes para
coletar uma massa de dados pessoais, o que pode comprometer a privacidade dos
dados dos usuários.

De acordo com~\textcite{russel2019}, existem diferentes categorias de
\acrshortpl{api} que os serviços podem fornecer aos desenvolvedores, como:

\begin{itemize}
    \item As~\acrshortpl{api} voltadas para o conteúdo fornecem acesso a
    dados publicados pelo serviço original.
    \item As~\acrshortpl{api} de recursos permitem que outros sites ou
    aplicativos móveis integrem o recurso existente de outro serviço em sua
    oferta.
    \item As~\acrshortpl{api} não oficiais destinam-se ao uso interno do
    serviço, mas também podem ser usadas por terceiros.
    \item As~\acrshortpl{api} de análise permitem que os desenvolvedores
    obtenham informações sobre os visitantes de seus sites
\end{itemize}

%\section{Autenticação} \label{sec:autenticacao}
%\subsection{Tipos de Autenticação}
%\label{subsec:tipos-de-autenticacao}
%\subsubsection{Autenticação de dois fatores}
%\subsubsection{Multifator}
%\subsection{Assinatura Digital}
%\label{subsec:assinatura-digital}
%\subsection{Normas para autenticação}
%\label{subsec:normas-para-autenticacao}