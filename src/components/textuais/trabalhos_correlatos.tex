\chapter{Trabalhos correlatos}\label{ch:trabalhos-correlatos}
%Neste capitulo serão apresentados trabalhos que correlacionam o
%uso de \acrshort{knn}, em implementações em Kotlin e/ou em outras
%linguagens.
As fontes serão apresentadas na Tabela \ref{tab:fontes-pesquisa}
\begin{table}[h!]
    \centering
    \begin{tabular}{|m{4cm}|m{4cm}|}
        \hline
        Nome da Fonte       & Link de Acesso               \\ \hline

        IEEExplore          & https://ieeexplore.ieee.org/ \\ \hline
        ACM Digital Library & https://dl.acm.org/          \\ \hline
        Springer            & https://link.springer.com/   \\ \hline
    \end{tabular}
    \caption[Fontes de pesquisa]{\textit{Fontes para pesquisa dos
    conteúdos academicos}}
    \label{tab:fontes-pesquisa}
\end{table}
Na busca dos artigos, foi realizado filtragem seguindo alguns
critérios de inclusão e exclusão, como, por exemplo, por meio do
ano de publicação, mantendo os mesmos entre 2017 à 2022, para que o
trabalho consista de materiais atualizados.
Os critérios de inclusão e exclusão são descritos sistematicamente
na tabela \ref{tab:criterios-exclusao}.
\newpage
\begin{table}[h!]
    \centering
    \begin{tabular}{|m{4cm}|m{4cm}|}
        \hline
        Critérios de inclusão & Critérios de Exclusão \\ \hline
        \begin{itemize}[leftmargin=10px]
            \item Artigos que estejam entre os anos de 2017 a 2023
            \item Artigos que contenham relações com documentos
            impressos
            \item Artigos que contenham relações com autenticação
            digital
            \item Artigos que disponham identificação de informa
            ções fraudulentas de documentos impressos
        \end{itemize} &
        \begin{itemize}[leftmargin=10px]
            \item Artigos anteriores à 2015.
        \end{itemize} \\ \hline
    \end{tabular}
    \caption[Critérios de inclusão e exclusão]{\textit{Critérios de
    inclusão e exclusão}}
    \label{tab:criterios-exclusao}
\end{table}

Conforme os dados apresentados na tabela \ref{tab:resultado-pesquisa},
na pesquisa através do site IEEE Xplore, foram encontrados 3 (três)
artigos com relevância para este trabalho dentre os 32 (trinta e
dois).

\begin{table}[h!]
    \centering
    \begin{tabular}{| m{3cm} | m{3cm} | m{3cm} |}
        \hline
        Local de Busca      & Resultado & Utilizados \\ \hline

        IEEExplore          & 32        & 3          \\ \hline
        ACM Digital Library & x2        & y2         \\ \hline
        Springer            & x3        & y3         \\ \hline
    \end{tabular}
    \caption[Resultado das pesquisas]{\textit{Resultado das
    pesquisas para utilização nos Trabalhos Correlatos.}}
    \label{tab:resultado-pesquisa}
\end{table}

\newpage
As \textit{strings} de busca utilizadas para pesquisa, estão
apresentadas na tabela \ref{tab:strings-busca}.
\begin{table}[h]
    \centering
    \begin{tabular}{|c|m{6cm}|l|l|}
        \hline
        Bibliotecas de pesquisa & \textit{Strings} de Pesquisa \\ \hline

        IEEE Xplore & ("All Metadata":authentication) AND ("All
        Metadata":printed document)
        \\ \hline
        ACM Digital Library &
        \\ \hline
        Springer &
        \\ \hline
    \end{tabular}
    \caption[Strings de busca]{\textit{Strings de busca utilizadas
    para adquirir os conteúdos}}
    \label{tab:strings-busca}
\end{table}