\newcommand{\hashfootnote}{
    \footnote{
        O algoritmo \textit{hash} é conhecido como uma função matemática
        criptográfica, na qual você possui dados de entrada e, após passar
        pela criptografia, eles apresentam valores de saída "padronizados", ou
        seja, as saídas devem possuir o mesmo tamanho (geralmente entre 128 e
        512 bits) e o mesmo número de caracteres alfanuméricos.
        Disponivel em :
        \url{https://www.voitto.com.br/blog/artigo/o-que-e-hash-e-como-funciona}
    }
}
\chapter{Trabalhos correlatos}\label{ch:trabalhos-correlatos}

Neste capítulo serão apresentados trabalhos que fazem parte da
estruturação da pesquisa, no qual foi de grande importancia para
obter informações para melhorar, resolver relacionados à mesma, e
para melhor entendimento da pesquisa pelo autor.
A tabela~\ref{tab:fontes-pesquisa} a seguir consta as fontes para a
busca dos trabalhos:

\begin{table}[h!]
    \centering
    \begin{tabular}{|m{4cm}|m{4.5cm}|}
        \hline
        Nome da Fonte       & Link de Acesso                \\ \hline
        IEEExplore          & https://ieeexplore.ieee.org   \\ \hline
        ACM Digital Library & https://dl.acm.org/           \\ \hline
        Science Direct      & https://www.sciencedirect.com \\ \hline
    \end{tabular}
    \caption[Fontes de pesquisa]{\textit{Fontes para pesquisa dos
    conteúdos acadêmicos}}
    \label{tab:fontes-pesquisa}
\end{table}

Na busca dos artigos, foi realizado filtragem seguindo alguns
critérios de inclusão e exclusão, no qual garantem que o trabalho
esteja atualizado, tenha consistência e garanta o embasamento
correto das informações aqui conferidas.

\begin{table}[h!]
    \centering
    \begin{tabularx}{\textwidth}{
        |>{\centering\arraybackslash}X
        |>{\centering\arraybackslash}X
        |}
        \hline
        \multicolumn{1}{|c|}{Critérios de inclusão} & \multicolumn{1}{c|}{
            Critérios de Exclusão} \\ \hline
        \begin{itemize}[leftmargin=10px]
            \item Artigos que estejam entre os anos de 2015 a 2023
            \item Artigos que contenham autenticação de documentos digitais,
            especificamente no formato \acrshort{pdf}
            \item Artigos que contenham relações com autenticação digital
        \end{itemize} &
        \begin{itemize}[leftmargin=10px]
            \item Artigos anteriores à 2015.
            \item Artigos que não possuam relação à autenticação de
            documentos digitais
            \item Artigos que tenham relação ao formato \acrshort{pdf}
        \end{itemize} \\ \hline
    \end{tabularx}
    \caption[Critérios de inclusão e exclusão]{\textit{Critérios de
    inclusão e exclusão}}
    \label{tab:criterios-exclusao}
\end{table}

Já a relação entre os dados obtidos de cada pesquisa, com suas
respectivas strings de busca, e os trabalhos utilizados, constam na
tabela~\ref{tab:resultado-pesquisa}, a seguir:

\begin{table}[h!]
    \newcommand{\ieeexplorer}{
        ("All Metadata": digital documents) AND ("All Metadata": authentication)
        AND (("All Metadata": system) OR ("All Metadata": API))
    }
    \newcommand{\acmdigital}{
        [All: "digital documents"] AND [All: "authentication"] AND [[All: "
        system"] OR [All: "API"]]
    }
    \newcommand{\sciencedirect}{
        ("digital documents" AND "authentication" AND ("system" OR "API"))
    }
    \centering
    \begin{tabularx}{\textwidth}{
        |>{\centering\arraybackslash}X
        |>{\centering\arraybackslash}X
        |>{\centering\arraybackslash}X
        |>{\centering\arraybackslash}X
        |}
        \cline{2-4}
        \multicolumn{1}{c|}{} & IEExplore & ACM Digital Library & ScienceDirect
        \\ \hline
        Strings de Busca & \ieeexplorer & \sciencedirect & \acmdigital
        \\ \hline
        Resultado & 110 & 51 & 65
        \\ \hline
        Utilizados & 2 & - & -
        \\ \hline
    \end{tabularx}
    \caption[Resultado das pesquisas]{Resultado das
    pesquisas para utilização nos Trabalhos Correlatos.}
    \label{tab:resultado-pesquisa}
\end{table}

Já na tabela~\ref{tab:trabalhos-correlatos}, segue com algums dados sobre os
trabalhos correlatos, como titulo, os escritores e o ano de publicação do mesmo.

\begin{table}[h!]
    \centering
    \begin{tabularx}{1\textwidth}{
        |>{\centering\arraybackslash}X
        |>{\centering\arraybackslash}X
        |>{\centering\arraybackslash}X|
    }
        \hline
        \multicolumn{1}{|c|}{Título} & Autores & Data de Publicação
        \\ \hline
        \multicolumn{3}{|c|}{IEEExplore} \\ \hline
        \citerow{ramadhan2023} \\ \hline
        \citerow{iman2021} \\ \hline
        \citerow{shree2022} \\ \hline
        \multicolumn{3}{|c|}{ACM Digital Library} \\ \hline
        \multicolumn{3}{|c|}{Science Direct} \\ \hline
    \end{tabularx}
    \caption[Informações dos Trabalhos Correlatos]{Informações referentes aos
    trabalhos selecionados}
    \label{tab:trabalhos-correlatos}
\end{table}

Sobre o trabalho de \textcite{ramadhan2023}, é falado a respeito de como a
pandemia de COVID-19 levou a um aumento significativo no uso de assinaturas
digitais em sistemas comerciais, especialmente na transferência de documentos
entre indivíduos por meios digitais.
Vários algoritmos para assinaturas digitais foram apresentados anteriormente;
no entanto, não se sabe qual algoritmo é mais adequado para documentos no
formato~\acrfull{pdf} em termos de segurança e velocidade de processamento.
Além disso, as soluções para serviços de assinatura digital atualmente
disponíveis são bastante caras.

Diante deste cenário,~\textcite{ramadhan2023} propõem uma análise de segurança,
consumo de memória e tempo de processamento de métodos de assinatura
digital em documentos~\acrshort{pdf}.
Além disso, os mesmos realizam o desenvolvimento de um protótipo de
assinatura digital para serviços~\textit{Web} que usa os melhores
algoritmos de assinatura digital analisados como resultado da análise.

Em relação ao trabalho de~\textcite{iman2021}, os mesmos falam em relação de
como atualmente, pelo crescimento exarcebado do setor de tecnologia da
informação, e o fácil acesso do mercado a ferramentas modernas para o
escritório, há uma demanda crescente por procedimentos de verificação e
autenticação de uma variedade de documentos cruciais, como documentos de
transações, bancários e oficiais, bem como certificações de instituições
educacionais.
Devido a isto, vários procedimentos difíceis e demorados tornaram a verificação
de documentos extremamente difícil e demorada.

Para resolver este problema,~\textcite{iman2021} proporam um aplicativo \textcite{Web}
descentralizado para verificação de documentos digitais no armazenamento em
nuvem~\acrshort{p2p} que aproveita a tecnologia \textit{blockchain} da
Ethereum para aprimorar o processo de verificação por meio de maior
transparência, abertura e auditabilidade.

O artigo, escrito por~\textcite{shree2022} fala que Tudo em nossa vida tem se
tornado digital.
Muitas de nossas tarefas manuais foram automatizadas, graças ao rápido avanço
da tecnologia.
Com isso também houve avanço no uso em documentações, levando as pessoas a
utilizarem documentos digitais.
Em contrapartida, o aumento de documentos falsos digitais ficaram mais
comuns; documentos importantes, como certificados educacionais e extratos
bancários, precisam ser verificados e autenticados.
Porém o processamento, verificação e autenticação desses documentos é complexo e
demorado.


Para isto,~\textcite{shree2022} trazem consigo um sistema \textit{Web}, que,
segundo os mesmos, passa em todos os critérios mencionados acima.
O sistema em si, armazena os documentos em nuvem, fazendo com que cada
arquivo possua um \textit{hash}\hashfootnote, assim fazendo com que a
verificação dos
documentos seja mais rapida e mais conveniente que o modelo tradicional.

