\newcommand{\hashfootnote}{
    \footnote{
        O algoritmo \textit{hash} é conhecido como uma função matemática
        criptográfica, na qual você possui dados de entrada e, após passar
        pela criptografia, eles apresentam valores de saída "padronizados", ou
        seja, as saídas devem possuir o mesmo tamanho (geralmente entre 128 e
        512 bits) e o mesmo número de caracteres alfanuméricos.
        Disponivel em :
        \url{https://www.voitto.com.br/blog/artigo/o-que-e-hash-e-como-funciona}
    }
}
\newcommand{\ieeexplorer}{
    ("All Metadata": digital documents) \\ AND ("All Metadata":
    authentication) \\
    AND (("All Metadata": system) \\ OR ("All Metadata": API))
}
\newcommand{\acmdigital}{
    [All: "digital documents"] \\ AND [All: "authentication"] \\ AND [[All: "
    system"] \\ OR [All: "API"]]
}
\newcommand{\springerlink}{
    'authentication AND "digital documents" AND (system OR API)'
}
\chapter{Trabalhos Correlatos}\label{ch:trabalhos-correlatos}

Neste capítulo serão apresentados trabalhos que fazem parte da
estruturação da pesquisa, no qual foi de grande importancia para
obter informações para melhorar, resolver relacionados à mesma, e
para melhor entendimento da pesquisa pelo autor.
A tabela~\ref{tab:fontes-pesquisa} a seguir consta as fontes para a
busca dos trabalhos:
\begin{table}[h!]
    \caption[Fontes de pesquisa]{Fontes de pesquisa dos conteúdos acadêmicos}
    \begin{tblr}{
        colspec={|c|l|},
        rowspec={ c c },
    }
        \hline
        Nome da Fonte       & \SetCell[c=1]{c}{Link de Acesso}  \\ \hline
        IEEExplore          & \url{https://ieeexplore.ieee.org} \\ \hline
        ACM Digital Library & \url{https://dl.acm.org/}         \\ \hline
        Springerlink        & \url{https://link.springer.com/}  \\ \hline
    \end{tblr}
    \sourcesearchfootnote
    \label{tab:fontes-pesquisa}
\end{table}

Também consta a representação gráfica da porcentagem dos resultados obtidos
através das pesquisas nas diferentes fontes, na figura~\ref{fig:grafico
-resultados-pesquisa}.

\begin{figure}[h!]
    \caption[Gráfico de resultados da pesquisa]{Gráfico de resultados de
    pesquisa}
    \begin{tikzpicture}
        \pie[radius=2.25, rotate=90]{48.67/IEEXplore, 28.76/ACM Digital
        Library, 22
        .56/Springerlink}
    \end{tikzpicture}
    \sourcesearchfootnote
    \label{fig:grafico-resultados-pesquisa}
\end{figure}

Na busca dos artigos, foi realizado filtragem seguindo alguns
critérios de inclusão e exclusão, no qual garantem que o trabalho
esteja atualizado, tenha consistência e garanta o embasamento
correto das informações aqui conferidas.
Os critérios estão informados na tabela~\ref{tab:criterios-exclusao}.

\begin{table}[h]
    \caption[Critérios de Inclusão e Exclusão]{Critérios de Inclusão e Exclusão}
    \begin{tblr}{
        colspec={|X[c]|X[c]|},
        rowspec={ c c },
        stretch=0,
        rows={ht=\baselineskip},
        cell{2}{1-2}={font=\small}
    }
        \hline
        Critérios de Inclusão & Critérios de Exclusão \\ \hline
        \begin{itemize}[leftmargin=10px]
            \item Artigos que estejam entre os anos de 2015 a 2023
            \item Artigos que contenham autenticação de documentos digitais,
            especificamente no formato \acrshort{pdf}
            \item Artigos que contenham relações com autenticação digital
        \end{itemize} &
        \begin{itemize}[leftmargin=10px]
            \item Artigos anteriores à 2015.
            \item Artigos que não possuam relação à autenticação de
            documentos digitais
            \item Artigos que tenham relação ao formato \acrshort{pdf}
        \end{itemize} \\ \hline
    \end{tblr}
    \sourcesearchfootnote
    \label{tab:criterios-exclusao}
\end{table}

Já a relação entre os dados obtidos de cada pesquisa, com suas
respectivas strings de busca, e os trabalhos utilizados, constam na
tabela~\ref{tab:resultado-pesquisa}, a seguir:

\begin{table}[h!]
    \centering
    \caption[Resultado das pesquisas]
    {Resultado das pesquisas para utilização nos Trabalhos Correlatos.}
    \begin{adjustbox}{width=0.9\columnwidth, center}
        \begin{tblr}{
            colspec={|X[c]|X|c|c|},
            rowspec={ m{1cm} m{1cm} m{1cm} m{1cm} },
            stretch=0,
            row{1}={font=\small},
            rows={ht=\baselineskip},
            cell{2-4}{1-2}={font=\scriptsize},
        }
            \hline
            & Strings de Busca & Resultado & Utilizados \\ \hline
            IEExplore           & \ieeexplorer     & 110       & 4          \\ \hline
            ACM Digital Library & \acmdigital      & 65        & 1          \\ \hline
            SpringerLink        & \springerlink    & 50        & 2          \\ \hline
        \end{tblr}
    \end{adjustbox}
    
    \sourcesearchfootnote
    \label{tab:resultado-pesquisa}
\end{table}

Na tabela~\ref{tab:trabalhos-correlatos}, segue com algums dados sobre os
trabalhos correlatos, como titulo, os escritores e o ano de publicação do mesmo.

\begin{table}[h!]
    \caption[Informações dos Trabalhos Correlatos]
    {Informações referentes aos trabalhos selecionados}
    \begin{tblr}{
        colspec = {|Q[10em]|c|c|},
        rowspec = { c c c },
        row{2,8,10} = {green9, font=\small},
        cell{3-7,9,11-Z}{1-3}={font=\scriptsize}
    }
        \hline
        Título & Autores & Data
        de Publicação \\ \hline
        \SetCell[c=3]{c}{IEEExplore} \\ \hline
        \citetitle*{ramadhan2023}    & \citeauthor*{ramadhan2023}    & \citedate{ramadhan2023}    \\ \hline
        \citetitle*{iman2021}        & \citeauthor*{iman2021}        & \citedate{iman2021}        \\ \hline
        \citetitle*{shree2022}       & \citeauthor*{shree2022}       & \citedate{shree2022}       \\ \hline
        \citetitle*{chakraborty2016} & \citeauthor*{chakraborty2016} & \citedate{chakraborty2016} \\ \hline
        \citetitle*{zachariah2016}   & \citeauthor*{zachariah2016}   & \citedate{zachariah2016}   \\ \hline
        \SetCell[c=3]{c}{ACM Digital Library} \\ \hline
        \citetitle*{singh2021}       & \citeauthor*{singh2021}       & \citedate{singh2021}       \\ \hline
        \SetCell[c=3]{c}{SpringerLink} \\ \hline
        \citetitle*{kabir2021}       & \citeauthor*{kabir2021}       & \citedate{kabir2021}       \\ \hline
        \citetitle*{wojtowicz2016}   & \citeauthor*{wojtowicz2016}   & \citedate{wojtowicz2016}   \\ \hline
    \end{tblr}
    \sourcesearchfootnote
    \label{tab:trabalhos-correlatos}
\end{table}

Sobre o trabalho de~\textcite{ramadhan2023}, é falado a respeito de como a
pandemia de COVID-19 levou a um aumento significativo no uso de assinaturas
digitais em sistemas comerciais, especialmente na transferência de documentos
entre indivíduos por meios digitais.
Vários algoritmos para assinaturas digitais foram apresentados anteriormente;
no entanto, não se sabe qual algoritmo é mais adequado para documentos no
formato~\acrfull{pdf} em termos de segurança e velocidade de processamento.
Além disso, as soluções para serviços de assinatura digital atualmente
disponíveis são bastante caras.

Diante deste cenário,~\textcite{ramadhan2023} propõem uma análise de segurança,
consumo de memória e tempo de processamento de métodos de assinatura
digital em documentos~\acrshort{pdf}.
Além disso, os mesmos realizam o desenvolvimento de um protótipo de
assinatura digital para serviços~\textit{Web} que usa os melhores
algoritmos de assinatura digital analisados como resultado da análise.

Em relação ao trabalho de~\textcite{iman2021}, os mesmos falam em relação de
como atualmente, pelo crescimento exarcebado do setor de tecnologia da
informação, e o fácil acesso do mercado a ferramentas modernas para o
escritório, há uma demanda crescente por procedimentos de verificação e
autenticação de uma variedade de documentos cruciais, como documentos de
transações, bancários e oficiais, bem como certificações de instituições
educacionais.
Devido a isto, vários procedimentos difíceis e demorados tornaram a verificação
de documentos extremamente difícil e demorada.

Para resolver este problema,~\textcite{iman2021} proporam um aplicativo \textcite{Web}
descentralizado para verificação de documentos digitais no armazenamento em
nuvem~\acrshort{p2p} que aproveita a tecnologia \textit{blockchain} da
Ethereum para aprimorar o processo de verificação por meio de maior
transparência, abertura e auditabilidade.

O artigo, escrito por~\textcite{shree2022} fala que Tudo em nossa vida tem se
tornado digital.
Muitas de nossas tarefas manuais foram automatizadas, graças ao rápido avanço
da tecnologia.
Com isso também houve avanço no uso em documentações, levando as pessoas a
utilizarem documentos digitais.
Em contrapartida, o aumento de documentos falsos digitais ficaram mais
comuns; documentos importantes, como certificados educacionais e extratos
bancários, precisam ser verificados e autenticados.
Porém o processamento, verificação e autenticação desses documentos é complexo e
demorado.

Para isto,~\textcite{shree2022} trazem consigo um sistema \textit{Web}, que,
segundo os mesmos, passa em todos os critérios mencionados acima.
O sistema em si, armazena os documentos em nuvem, fazendo com que cada
arquivo possua um \textit{hash}\hashfootnote, assim fazendo com que a
verificação dos
documentos seja mais rapida e mais conveniente que o modelo tradicional.

A substituição de documentos em papel por documentos eletrônicos está se
tornando
cada vez mais comum no mundo moderno, com o surgimento do \textit{e-business},
\textit{e-government} e \textit{e-commerce}.
No entanto, a adulteração de informações e a falsificação de assinaturas têm
aumentado de forma impressionante.
Qualquer pessoa que assine um documento digitalmente, seja uma empresa ou um
usuário individual, pode se tornar vítima de crimes digitais\cite{
    chakraborty2016}.

Para este cenário~\textcite{chakraborty2016} propôs uma plataforma de
assinatura digital \textit{cloud}, habilitada para autenticação biométrica, que
cria uma solução de segurança aprimorada para a indústria de criptografia.
A proposta envolveu a criação e validação de assinaturas digitais usando um
\textit{framework} baseado em nuvem.
Similar à~\acrfull{pki}, cada usuário possui um conjunto de duas chaves: uma
chave privada e uma chave pública.
Para acessar a chave privada e assinar um documento, o signatário deve validar
sua identidade biométrica (padrão de veias) e realizar uma verificação única
da chave.
Antes de ter sua assinatura ou documento verificado, o destinatário deve
utilizar
adicionalmente seu padrão de veias para estabelecer sua identidade.
O destinatário precisa fornecer ao sistema a chave única compartilhada e a chave
pública do signatário para concluir o procedimento de verificação.

Para~\textcite{zachariah2016} obter documentos importantes assinados inicialmente
era uma tarefa estressante, especialmente se o signatário não estivesse presente.
Neste caso, o único método para obter a assinatura nos documentos seria enviá-los
pelo correio, aguardar que os assinassem e, em seguida, enviar o documento de volta.
Então, as assinaturas digitais entraram em cena, tornando todo o procedimento
simples e acessível.
Uma desvantagem desse método é que os documentos assinados com assinaturas
digitais são salvos em cópia eletrônica, enquanto muitos documentos legais e
oficiais (como certidões de nascimento, certificados de seguro, decretos de
divórcio, acordos de divisão de propriedade, passaportes e carteiras de
motorista) ainda precisam ser apresentados e assinados em cópia física\cite{zachariah2016}.

\textcite{zachariah2016} sugere uma maneira pela qual, utilizando~\acrfull{iot},
uma pessoa possa assinar um documento em um local específico e ter essa assinatura
refletida simultaneamente em um documento similar em outro lugar, em tempo real.
O usuário principal (autenticador) utiliza uma caneta digital para assinar o
documento em papel.

A área de assinatura do documento contém um padrão de pontos específico, que
auxilia a caneta digital a rastrear os movimentos conforme a assinatura é criada.
Enquanto o papel está sendo assinado, a caneta digital também registra a
impressão digital do usuário para posterior verificação.
As informações obtidas são então convertidas em \textit{bits} binários e
codificadas usando criptografia de DNA.
Esses dados criptografados são transmitidos ao usuário secundário (receptor),
onde é decodificado e impresso na segunda cópia do documento.
Assim, a assinatura é transferida de forma segura de um papel para o outro\cite{zachariah2016}.

Segundo o trabalho de~\textcite{singh2021} a questão da segurança de dados
digitais via Internet tem crescido astronomicamente nos últimos anos como
resultado da crescente popularidade da Internet.
A marca d'água é usada para proteger dados digitais de indivíduos não
autorizados em uma variedade de aplicativos.
Para isso,~\textcite{singh2021} propôs uma técnica de marca d'água baseada em
criptografia e compactação conjunta para a segurança de documentos digitais.
Essa tecnologia fornece uma ferramenta para confidencialidade do sistema,
proteção de direitos autorais e excelente desempenho de compactação.
O método proposto por~\textcite{singh2021} consiste em três etapas principais:
incorporação de várias marcas d'água usando uma transformada de contorno sem
subamostragem, uma transformada de wavelet discreta redundante e decomposição
de valor singular; criptografia e compactação usando SHA-256 e~\acrfull{lzw},
respectivamente; e extração/recuperação de várias marcas d'água de uma imagem
de cobertura potencialmente distorcida.

Conforme dito por~\textcite{kabir2021} devido aos profundos avanços nas
tecnologias de internet e comunicação digital,
os processos de geração de dados estão passando por rápidas transformações na
sociedade moderna.
Atualmente, o sistema de comunicação digital \textit{on-line} desempenha um
papel
crucial ao facilitar o armazenamento e a disseminação de arquivos multimídia,
incluindo imagens, áudio e vídeos.
No entanto, durante a transmissão e o armazenamento de multimídia,
os dados podem sofrer modificações que os tornam suscetíveis à exploração
não autorizada por intrusos\cite{kabir2021}.

Diante este cenário,~\textcite{kabir2021} apresenta uma análise da
suscetibilidade de documentos digitais e um método para proteger e detectar
alterações não autorizadas de dados multimídia.
O método proposto por~\textcite{kabir2021} foi um algoritmo de marca d'água,
baseado em um algoritmo de compressão com perdas, a fim de garantir a
autenticação e detectar falsificações.

A crescente quantidade e qualidade dos sensores integrados, a capacidade de
computação
dos dispositivos e a conectividade sem fio permitem aproveitar a autenticação
biométrica contínua em dispositivos móveis.
No entanto, o ambiente do usuário pode reduzir a precisão do reconhecimento
biométrico ou tornar o procedimento de aquisição desagradável para um usuário
móvel em um determinado momento, portanto, a utilidade real do aplicativo de
autenticação biométrica depende do contexto de uso\cite{wojtowicz2016}.

\textcite{wojtowicz2016} apresentam um modelo de autenticação biométrica baseado
em contexto para dispositivos móveis.
O modelo genérico projetado e verificado com uma implementação de prova de
conceito
serve como base para a construção de sistemas de autenticação móvel adaptáveis e
extensíveis, dependentes do contexto.