\section{Quadro relacional}
Neste capitulo descrevem-se as metodologias utilizadas em cada etapa dos objetivos propostos para este artigo.

\newcommand{\definicaoTema}{
\item Analisado os temas propostos com o orientador, definindo a utilidade e necessidade do mesmo.
}

\newcommand{\buscaArtigos}{
    \item Após a delimitação do tema, foi iniciado buscas para referencial bibliográfico, com conceitos como: Assinatura digital, Autenticação de dois fatores, entre outros.
}

\newcommand{\filtragemArtigos}{
\item Posteriormente, foi realizado filtragem para manter consistência nos artigos referenciais, de forma a não poluir o trabalho com conteúdo desnecessário.
}

\newcommand{\definicaoEspecificacoes}{
 \item Por conseguinte, foi definido as especificações para o uso da \acrshort{api}, no contexto do \acrfull{ies}.
}

\newcommand{\implementacaoAPI}{
    \item Com a definição das especificações, foi possivel realizar a implementação do sistema, visando entregar o produto em estágio de testes, como \acrfull{mvp}.
}

\newcommand{\testeAPI}{
    \item Por fim, após a bateria de testes técnicos, o \acrshort{mvp} foi testado por um usuário para garantir que a \acrshort{api} como um todo está funcionando conforme as especificações.
}

\begin{enumerate}[label=\arabic*\textdegree\space Etapa:, leftmargin=2cm]
    \definicaoTema
    \buscaArtigos
    \filtragemArtigos
    \definicaoEspecificacoes
    \implementacaoAPI
    \testeAPI
\end{enumerate}

\begin{longtable}[hp]{|m{3cm}|b{140px}|}
\hline
\centering
Objetivo Específico & Etapa Metodológica\\\hline
\endfirsthead
\endhead 

\buscaReferencia & 
\begin{tabular}[c]{@{}m{4.6cm}@{}}
    \begin{itemize}[leftmargin=10px]
        \definicaoTema
        \buscaArtigos
        \filtragemArtigos
    \end{itemize}
\end{tabular} \\ \hline

\arquitetura & 
\begin{tabular}[c]{@{}m{4.6cm}@{}}
    \begin{itemize}[leftmargin=10px]
        \definicaoEspecificacoes
    \end{itemize}
\end{tabular} \\ \hline

\implementacao &
\begin{tabular}[c]{@{}m{4.6cm}@{}}
    \begin{itemize}[leftmargin=10px]
        \implementacaoAPI
    \end{itemize} 
\end{tabular} \\ \hline

\testes &
\begin{tabular}[c]{@{}m{4.6cm}@{}}
    \begin{itemize}[leftmargin=10px]
        \testeAPI
    \end{itemize} 
\end{tabular} \\ \hline

\caption[Aspectos metodológicos e objetivos específicos]{\textit{Relação entre os aspectos metodológicos e os objetivos específicos}}
\label{table:aspectosmetodologicos}
\end{longtable}