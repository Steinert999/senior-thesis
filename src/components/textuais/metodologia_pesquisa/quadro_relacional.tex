\section{Procedimentos Metotodológicos}\label{sec:quadro-relacional}

Neste capitulo descrevem-se as metodologias utilizadas em cada
etapa dos objetivos propostos para este artigo, assim definido na
tabela \ref{tab:aspectos-metodologicos}.
\newcommand{\definicaoTema}{
    \item Analisado os temas propostos com o orientador, definindo
    a utilidade e necessidade do mesmo.
}
\newcommand{\buscaArtigos}{
    \item Após a delimitação do tema, foi iniciado buscas para
    referencial bibliográfico, com conceitos como: Assinatura
    digital, Autenticação de dois fatores, entre outros.
}
\newcommand{\filtragemArtigos}{
    \item Posteriormente, foi realizado filtragem para manter
    consistência nos artigos referenciais, de forma a não poluir o
    trabalho com conteúdo desnecessário.
}
\newcommand{\definicaoEspecificacoes}{
    \item Por conseguinte, foi definido as especificações para o
    uso da \acrshort{api}, no contexto do \acrfull{iesgf}.
}
\newcommand{\implementacaoAPI}{
    \item Com a definição das especificações, foi possivel realizar
    a implementação do sistema, visando entregar o produto em est ágio
    de testes, como \acrfull{mvp}.
}
\newcommand{\testeAPI}{
    \item Por fim, após a bateria de testes técnicos, o \acrshort{mvp}
    foi testado por um usuário para garantir que a \acrshort{api}
    como um todo está funcionando conforme as especificações.
}
\begin{enumerate}[label=\arabic*\textdegree\space Etapa:,
    leftmargin=2cm]
    \definicaoTema
    \buscaArtigos
    \filtragemArtigos
    \definicaoEspecificacoes
    \implementacaoAPI
    \testeAPI
\end{enumerate}
\section{Quadro Metodológico}\label{sec:quadro-metodologico}

Na tabela~\ref{tab:aspectos-metodologicos} foi evidenciado com base no
tópico de objetivos específicos juntamente com os procedimentos
metodológicos a relação dos mesmos, conforme a seguir:

\begin{table}[h!]
    \newcounter{rowno}
    \setcounter{rowno}{0}
    \centering
    \begin{tabular}{|m{150px}|
            >{\stepcounter{rowno}\therowno\textdegree\space Etapa}c|}
        \hline
        \multicolumn{1}{|c|}{Objetivos Especificos} &
        \multicolumn{1}{c|}{Etapas} \\ \hline
        \multirow{3}{150px}{\buscaReferencia} & \\ \cline{2-2}
        & \\ \cline{2-2}
        & \\ \hline
        \arquitetura                          & \\ \hline
        \implementacao                        & \\ \hline
        \testes                               & \\ \hline
    \end{tabular}
    \caption[Aspectos metodológicos e objetivos específicos]{
        \textit{Relação entre os aspectos metodológicos e os
        objetivos específicos}}
    \label{tab:aspectos-metodologicos}
\end{table}