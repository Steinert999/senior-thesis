\chapter{Metodologia da Pesquisa}\label{ch:metodologia-da-pesquisa}

Este capitulo contém informações sobre a composição das
características metodológicas deste trabalho, a partir de conceitos e
descrições analíticas sobre metodologia.

Conforme \textcite{wazlawick2009}, o mesmo define o termo
“metodologia da pesquisa” como o estudo dos métodos, onde embora
seja usado comumente em trabalhos científicos, o melhor termo que
se adéqua a ideia é o de “método de pesquisa”.
Onde o mesmo define como método de pesquisa uma sequência de passos
necessários para demonstrar que o objetivo proposto foi atingido.

Segundo \textcite{gil2017} a ciência, como outras formas de
conhecimento, tem o objetivo central de chegar a veracidade dos
fatos, porém o que a difere das demais é a sua característica
fundamental de verificabilidade.
Para um conhecimento poder se tornar cientifico é necessário
identificar as técnicas que possibilitam a sua verificação, ou seja,
determinar o método que possibilitou chegar a esse conhecimento.
Desta forma, o método pode ser definido como o caminho para esse fim
e método cientifico como um conjunto de procedimentos intelectuais
e técnicos para atingir certo conhecimento \cite{gil2017}.
% Aplicada
Segundo a expectativa deste trabalho, que é desenvolver um sistema
que possa auxiliar os professores e alunos, facilitando-os a manter
a autencidade dos trabalhos envolvidos, define-se por isso o mesmo
como uma pesquisa "aplicada", onde, dito por \textcite{gil2017} que
quando a pesquisa pretende resolver problemas ou aprimorar processos
em vez de apenas acumular conhecimento, diz-se que ela é "aplicada"
e também por \textcite{prodanov2013} que o objetivo da pesquisa
aplicada é produzir informações que possam ser usadas no mundo real
para tratar de questões específicas.
% Bibliográfica
De forma que, as informações contidas neste trabalho, estão paralelas
às informações obtidas em artigos, documentos e livros correlacionados
ao mesmo, este artigo encontra-se definido como bibliográfico \cite{prodanov2013}.
A principal característica, como afirma \textcite{gil2017}, é que o
pesquisador tem uma vasta gama de opções e deve estar consciente das
chances de não comprometer a pesquisa, especificamente devido a essa
quantidade de material prontamente disponível de forma acessível.
% Qualitativa
O projeto é definido como qualitativo, pois há relação direta entre
a aplicação desenvolvida e o mundo real,
conforme dito por \textcite{prodanov2013}:

\begin{sidecite}
    ¨Pesquisa qualitativa: considera que há uma relação dinâmica
    entre o mundo real e o sujeito, isto é, um vínculo indissociável
    entre o mundo objetivo e a subjetividade do sujeito que não pode
    ser traduzido em números.
    A interpretação dos fenômenos e a atribuição de significados são
    básicas no processo de pesquisa qualitativa.
    Esta não requer o uso de métodos e técnicas estatísticas.¨\cite[p.~70]{prodanov2013}
\end{sidecite}

% Exploratória
De acordo com \textcite{prodanov2013} a pesquisa exploratória tem
como finalidade proporcionar mais informações sobre o assunto que
vamos investigar, possibilitando sua definição e seu delineamento,
isto é, facilitar a delimitação do tema da pesquisa; orientar a
fixação dos objetivos e a formulação das hipóteses ou descobrir um
novo tipo de enfoque para o assunto.
Em concordância com o que foi dito por \textcite{prodanov2013}, a
pesquisa tem por definição a descoberta de um enfoque para o assunto
relacionado.

\newcommand{\footnoteendtoend}{
    \footnote{
        O teste \textit{end-to-end} é uma metodologia utilizada para testar se
        o fluxo de um aplicativo está sendo executado conforme o projeto do
        início ao fim.
        Disponivel em: \url{https://www.cedrotech.com/blog/teste-end-to-end/}
    }
}
\section{Procedimentos Metotodológicos}\label{sec:quadro-relacional}

Neste capitulo descrevem-se as metodologias utilizadas em cada
etapa dos objetivos propostos para este artigo, assim definido na
tabela~\ref{tab:aspectos-metodologicos}.
\newcommand{\definicaoTema}{
    \item
}
\newcommand{\buscaArtigos}{
    
    \item Busca de referencial bibliográfico, com conceitos como:
    Assinatura digital, Autenticação de dois fatores, entre outros.
}
\newcommand{\filtragemArtigos}{
    \item Filtragem para manter consistência nos artigos referenciais,
    para não poluir o trabalho com conteúdo desnecessário.
}
\newcommand{\definicaoEspecificacoes}{
    \item Definição das regras de negócio, estipuladas em conjunto com os
    responsáveis
    pelo curso bacharel de ciência da computação da instituição.
}
\newcommand{\implementacaoAPI}{
    \item Implementação do sistema, visando entregar o produto em estágio
    de testes, como~\acrfull{mvp}.
}
\newcommand{\testeAPI}{
    \item Testes \textit{end-to-end}\footnoteendtoend, realizados por um
    voluntário,
    fornecido pelo curso de ciência da computação, para garantir que a
    ~\acrshort{api} está funcionando conforme as especificações.
}
\begin{enumerate}[label=\arabic*\textdegree\space Etapa:, leftmargin=2cm]
    \item Levantamento de fontes confiáveis, livros, artigos acadêmicos
    e recursos online que abordem a autenticação, principalmente relacionada à
    autenticação biométrica com assinatura digital em sistemas de computação.
    \item Realizar reuniões com representantes do curso de ciência da computação
    para entender as necessidades específicas do sistema.
    \item Utilizar ferramentas de modelagem para esboçar a arquitetura do
    sistema,
    considerando requisito como escalabilidade, segurança e integração.
    \item Apresentar a arquitetura proposta para feedback e revisão pelos
    especialistas do curso,
    garantindo alinhamento com as necessidades estipuladas.
    \item Identificar os \textit{endpoints}, métodos e funcionalidades que
    a~\acrshort{api}deve oferecer, baseando-se na arquitetura definida.
    \item Desenvolver a~\acrshort{api}, integrando-a com a~\acrshort{api} de
    assinatura eletrônica fornecida pelo~\citeauthor*{govbr2020}.
    \item Realizar testes para garantir que cada parte da~\acrshort{api}
    funcione conforme o esperado.
    \item Criar um ambiente simulado que represente o ambiente de produção,
    com condições controladas para testes.
    \item Realizar testes que simulem interações da~\acrshort{api} com outros
    sistemas
    ou módulos, verificando a interoperabilidade.
    \item Envolver usuários do curso de ciência da computação para testar a
    ~\acrshort{api} em um cenário real, coletando feedback e realizando
    ajustes conforme necessário.
\end{enumerate}
\section{Quadro Metodológico}\label{sec:quadro-metodologico}

Na tabela~\ref{tab:aspectos-metodologicos} foi evidenciado com base no
tópico de objetivos específicos juntamente com os procedimentos
metodológicos a relação dos mesmos, conforme a seguir:

\newcounter{rowno}
\newcommand{\etp}{\stepcounter{rowno}\arabic{rowno}\textdegree\space Etapa}
\newcommand{\resultBuscaReferencia}{
    Documentação consolidada que explique os conceitos fundamentais de autenticação,
    suas vertentes e aplicações em sistemas computacionais.
}
\newcommand{\resultArquitetura}{
    Documento detalhado da arquitetura do sistema, incluindo diagramas,
    especificações técnicas e requisitos atendidos, aprovado pelos especialistas
    do curso.
}
\newcommand{\resultImplementacao}{
    \acrshort{api} funcional e implementada conforme as especificações da
    arquitetura, pronta para ser integrada ao sistema.
}
\newcommand{\resultTestes}{
    Validação positiva da~\acrshort{api} no ambiente simulado, com o usuário do
    curso reportando um uso sem problemas e confirmando a conformidade com as
    especificações estabelecidas.
}
\begin{table}[h!]
    \caption[Aspectos metodológicos e objetivos específicos]{Relação entre os
    aspectos metodológicos e os objetivos específicos}
    \begin{tblr}{
        colspec={|X|c|X|},
        rowspec={ c c c },
        cell{2-11}{1,3}={font=\scriptsize},
        cell{3}{1,3}={r=4}{c},
        cell{7}{1,3}={r=2}{c},
        cell{9}{1,3}={r=3}{c}
    }
        \hline
        Objetivos Específicos & Etapas & Resultados
        \\ \hline
        
        \buscaReferencia & \etp & \resultBuscaReferencia
        \\ \hline
        \arquitetura   & \etp & \resultArquitetura   \\ \hline
        & \etp &                      \\ \hline
        & \etp &                      \\ \hline
        & \etp &                      \\ \hline
        \implementacao & \etp & \resultImplementacao \\ \hline
        & \etp &                      \\ \hline
        \testes        & \etp & \resultTestes        \\ \hline
        & \etp &                      \\ \hline
        & \etp &                      \\ \hline
    \end{tblr}
    \sourcesearchfootnote
    \label{tab:aspectos-metodologicos}
\end{table}

