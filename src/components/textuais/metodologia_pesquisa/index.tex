\chapter{Metodologia da Pesquisa}

Este capitulo contém informações sobre a composição das características metodológicas deste trabalho, a partir de conceitos e descrições analíticas sobre metodologia.

Conforme \textcite{wazlawick2009}, o mesmo define o termo “metodologia da pesquisa” como o estudo dos métodos, onde embora seja usado comumente em trabalhos científicos, o melhor termo que se adéqua a ideia é o de “método de pesquisa”. Onde o mesmo define como método de pesquisa uma sequência de passos necessários para demonstrar que o objetivo proposto foi atingido.

Segundo \textcite{gil2017} a ciência, como outras formas de conhecimento, tem o objetivo central de chegar a veracidade dos fatos, porém o que a difere das demais é a sua característica fundamental de verificabilidade. Para um conhecimento poder se tornar cientifico é necessário identificar as técnicas que possibilitam a sua verificação, ou seja, determinar o método que possibilitou chegar a esse conhecimento. Desta forma, o método pode ser definido como o caminho para esse fim e método cientifico como um conjunto de procedimentos intelectuais e técnicos para atingir certo conhecimento \cite{gil2017}.

% Aplicada

Segundo a expectativa deste trabalho, que é desenvolver um sistema que possa auxiliar os professores e alunos, facilitando-os a manter a autencidade dos trabalhos envolvidos, define-se por isso o mesmo como uma pesquisa "aplicada", onde, dito por \textcite{gil2017} que quando a pesquisa pretende resolver problemas ou aprimorar processos em vez de apenas acumular conhecimento, diz-se que ela é "aplicada" e também por \textcite{prodanov2013} que o objetivo da pesquisa aplicada é produzir informações que possam ser usadas no mundo real para tratar de questões específicas.
% Bibliográfica

De forma que, as informações contidas neste trabalho, estão paralelas às informações obtidas em artigos, documentos e livros correlacionados ao mesmo, este artigo encontra-se definido como bibliográfico \cite{prodanov2013}. A principal característica, como afirma \textcite{gil2017}, é que o pesquisador tem uma vasta gama de opções e deve estar consciente das chances de não comprometer a pesquisa, especificamente devido a essa quantidade de material prontamente disponível de forma acessível.
% Qualitativa
O projeto é definido como qualitativo, pois há relação direta entre a aplicação desenvolvida e o mundo real, conforme dito por \textcite{prodanov2013}:

\begin{fullcite}
    ¨Pesquisa qualitativa: considera que há uma relação dinâmica entre o mundo
real e o sujeito, isto é, um vínculo indissociável entre o mundo objetivo
e a subjetividade do sujeito que não pode ser traduzido em números. A
interpretação dos fenômenos e a atribuição de significados são básicas
no processo de pesquisa qualitativa. Esta não requer o uso de métodos
e técnicas estatísticas.¨\cite[p.~70]{prodanov2013}   
\end{fullcite}

% Exploratória

\newcommand{\footnoteendtoend}{
    \footnote{
        O teste \textit{end-to-end} é uma metodologia utilizada para testar se
        o fluxo de um aplicativo está sendo executado conforme o projeto do
        início ao fim.
        Disponivel em: \url{https://www.cedrotech.com/blog/teste-end-to-end/}
    }
}
\section{Procedimentos Metotodológicos}\label{sec:quadro-relacional}

Neste capitulo descrevem-se as metodologias utilizadas em cada
etapa dos objetivos propostos para este artigo, assim definido na
tabela~\ref{tab:aspectos-metodologicos}.
\newcommand{\definicaoTema}{
    \item Análise do tema propostos com o orientador, definindo
    a utilidade e necessidade do mesmo.
}
\newcommand{\buscaArtigos}{
    \item Busca de referencial bibliográfico, com conceitos como:
    Assinatura digital, Autenticação de dois fatores, entre outros.
}
\newcommand{\filtragemArtigos}{
    \item Filtragem para manter consistência nos artigos referenciais,
    para não poluir o trabalho com conteúdo desnecessário.
}
\newcommand{\definicaoEspecificacoes}{
    \item Definição das regras de negocio, estipuladas em conjunto com o
    bacharel de ciência da computação da instituição.
}
\newcommand{\implementacaoAPI}{
    \item Implementação do sistema, visando entregar o produto em estágio
    de testes, como~\acrfull{mvp}.
}
\newcommand{\testeAPI}{
    \item Testes \textit{end-to-end}\footnoteendtoend, realizados por um voluntário,
    fornecido pelo bacharel de ciência da computação, para garantir que a
    ~\acrshort{api} está funcionando conforme as especificações.
}
\begin{enumerate}[label=\arabic*\textdegree\space Etapa:,
    leftmargin=2cm]
    \definicaoTema
    \buscaArtigos
    \filtragemArtigos
    \definicaoEspecificacoes
    \implementacaoAPI
    \testeAPI
\end{enumerate}
\section{Quadro Metodológico}\label{sec:quadro-metodologico}

Na tabela~\ref{tab:aspectos-metodologicos} foi evidenciado com base no
tópico de objetivos específicos juntamente com os procedimentos
metodológicos a relação dos mesmos, conforme a seguir:

\newcounter{rowno}
\newcommand{\etp}{\stepcounter{rowno}\arabic{rowno}\textdegree\space Etapa}
\begin{table}
    \caption[Aspectos metodológicos e objetivos específicos]{Relação entre os
    aspectos metodológicos e os objetivos específicos}
    \begin{tblr}{|Q[c, 15em]|c|}
        \hline
        Objetivos Específicos              & Etapas \\ \hline
        \SetCell[r=3]{m}{\buscaReferencia} & \etp   \\ \hline
        & \etp   \\ \cline{2-2}
        & \etp   \\ \hline
        \arquitetura                       & \etp   \\ \hline
        \implementacao                     & \etp   \\ \hline
        \testes                            & \etp   \\ \hline
    \end{tblr}
    \sourcesearchfootnote
    \label{tab:aspectos-metodologicos}
\end{table}