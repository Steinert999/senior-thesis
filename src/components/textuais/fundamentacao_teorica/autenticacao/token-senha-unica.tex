\subsection{Autenticação via \texit{Token} de Senha Única}
\label{subsec:autenticacao-token-senha-unica}

Os \textit{tokens} de senha única ou~\acrfull{otp} são senhas dinâmicas criadas
por um sistema gerador, no qual são capazes serem utilizadas apenas uma vez e
por um período limitado, restringindo a janela de oportunidade para
um atacante agir\cite{christiana2019}.

Segundo~\textcite{huiyi2013}, a geração de~\acrshort{otp} pode ser realizada
por meio de métodos de hash como~\acrfull{sha1} ou~\acrfull{md5}, assim como por
meio de inteiros aleatórios ou contadores que aumentam a cada uso, a fim de
sincronizar o tempo entre o usuário e o servidor.

Os sistemas~\acrshort{otp} podem ser entendidos como uma ponte entre um
mecanismo de autenticação mais seguro e a autenticação estática por senha.
Isso facilita a migração de aplicativos legados, como unidades centrais de
processamento (\textit{mainframes}) e sites, que foram originalmente projetados
para funcionar apenas com senhas\cite{idrus2013}.

Porém, os sistemas~\acrshort{otp} contém algumas vulnerabilidades também.
Segundo~\textcite{hoyul2015} existem certas vulnerabilidades no~\acrshort{otp
} que o tornam
suscetível a ataques~\acrfull{mitm} e ataques~\acrfull{mitpcp1}.
Durante a fase de inicialização do algoritmo~\acrshort{otp}, um servidor e os
clientes trocam certos dados confidenciais.
Um invasor pode efetivamente se autenticar se ele usar um ataque~\acrshort{mitm}
para obter algumas informações confidenciais.
Um invasor também pode obter acesso à memória do dispositivo e
produzir valores~\acrshort{otp} se ele implantar código malicioso nos
dispositivos dos usuários\cite{hoyul2015}.