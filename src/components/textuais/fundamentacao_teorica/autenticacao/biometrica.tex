\subsection{Autenticação Biométrica}
\label{subsec:autenticacao-biometrica}

A biometria é uma técnica utilizada na ciência da computação para controlar o
acesso e gerenciar a identidade de usuários.
O termo biometria tem origem no grego antigo sendo composto pelos termos \textit{bios},
que significa vida, e \textit{metron}, que se refere à medição.
A ciência da biometria estuda como identificar indivíduos por meio de suas
características físicas ou comportamentais\cite{idrus2013}.

Portanto, a autenticação biométrica ou biometria é o processo pelo qual
computadores
são utilizados para identificar indivíduos, apesar das diferenças existentes
de cada um.
Cada técnica biométrica é incapaz de determinar a identidade "verdadeira" de um
indivíduo.
Em vez disso, a identificação de uma pessoa e quaisquer características pessoais
fornecidas no momento da inscrição no sistema são as únicas informações que a
tecnologia biométrica pode associar a elas, além de um padrão biométrico\cite{wayman2005}.

A natureza humana faz com que as pessoas confiem naturalmente nas características
físicas do corpo humano que são visíveis, como as feições do rosto, a voz, o
caminhar,
a assinatura, entre outros.
Como esses traços são particulares de cada indivíduo, esses métodos são
excelentes para identificar outras pessoas\cite{alsaadi2015}.

\begin{figure}[h!]
    \centering
    \caption[Classificação geral dos sistemas biométricos]{Classificação
    geral dos sistemas biométricos}
    \begin{tikzpicture}[
        action/.style={draw, very thick, rounded corners,
        text width=2.5cm,
        align=center}
    ]
        \node[action](main){Autenticação baseada em biometria};
        \node[action](physiological)[below left=1cm and 0.1cm of main]
        {Fisiológico};
        \node[action](behavioral)[below right=1cm and 0.1cm of main]
        {Comportamental};
        \node(p-childs)[below=1cm of physiological]{
            \begin{tblr}{|c|}
                \hline
                Reconhecimento de faces   \\ \hline
                Geometria da mão          \\ \hline
                Impressão digital         \\ \hline
                Íris                      \\ \hline
                Retina                    \\ \hline
                Reconhecimento de orelhas \\ \hline
            \end{tblr}
        }
        \node(b-childs)[below=1cm of behavioral]{
            \begin{tblr}{|c|}
                \hline
                Voz            \\ \hline
                Assinatura     \\ \hline
                Marcha         \\ \hline
                Toque de tecla \\ \hline
                Dinâmica       \\ \hline
                Lábios         \\ \hline
                Movimentos     \\ \hline
            \end{tblr}
        }
        
        \draw[thick] (main.south) -| (0, -1) edge[->] (physiological.north);
        \draw[thick] (main.south) -| (0, -1) edge[->] (behavioral.north);
        \draw[thick, ->](physiological) edge (p-childs);
        \draw[thick, ->](behavioral) edge (b-childs);
    \end{tikzpicture}
    \floatfoot{Adaptado de \cite{alsaadi2015}}
\end{figure}
\clearpage
A principal vantagem dessas técnicas de autenticação é a estreita conexão que
existe entre o usuário e seu autenticador, que é o dado biométrico.
Além disso, ao contrário da maioria das outras técnicas de autenticação,
é difícil replicar as características biométricas de um indivíduo.
No entanto, a autenticação biométrica tem uma desvantagem, que é a imprevisibilidade
do resultado de verificação.
Por exemplo, na autenticação por impressão digital, um erro pode ocorrer devido
ao alinhamento inadequado do dedo\cite{idrus2013}.