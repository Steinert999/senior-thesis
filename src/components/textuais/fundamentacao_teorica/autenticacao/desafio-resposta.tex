\subsection{Autenticação baseada em desafio-resposta}
\label{subsec:autenticacao-desafio-resposta}

O~\acrfull{cram}, ou mecanismo de autenticação desafio-resposta, é um
procedimento
cujo objetivo é verificar a autenticidade de uma entidade que solicita acesso,
exigindo que ela demonstre estar ciente de um segredo, sem revelá-lo ao
verificador.
O procedimento consiste no verificador oferecendo um desafio ao solicitante a
partir de um conjunto de desafios preparados.
O solicitante recebe um desafio que varia ao longo do tempo do verificador e,
em resposta, utiliza uma função dependente do segredo para gerar uma resposta
Caso o solicitante responda corretamente ao desafio oferecido, ele é autenticado
e recebe acesso a um computador, rede ou outro recurso\cite{gilad2013}.

\begin{figure}[h!]
    \centering
    \caption[Diagrama representação autenticação via desafio-resposta]
    {Diagrama de transição de estados representando autenticação com
    desafio-resposta}
    \begin{tikzpicture}
    [
        action/.style={draw, very thick, rounded corners, text width=1.5cm,
        align=center},
        initial/.style={draw, circle, fill, minimum size=10mm},
        final/.style={draw, double, thick, circle, fill, minimum size=10mm},
        decision/.style={draw, diamond, black, fill, minimum size=10mm},
    ]
        \node[initial](begin);
        \node[action](request)[right=0.5cm of begin]
        {Solicitação de autenticação.};
        \node[action](challenge)[right=0.5cm of request]
        {Desafio enviado ao usuário.};
        \node[decision, label={
            [text width=2.5cm, align=center]
            O usuário tenta fazer o desafio, a resposta está correta?}]
        (attempt)[right=0.75cm of challenge];
        \node[action](success)[right=1.25cm of attempt]{Acesso garantido.};
        \node[final](end)[right=0.5cm of success];
        
        \draw[thick, ->](begin) edge (request);
        \draw[thick, ->](request) edge (challenge);
        \draw[thick, ->](challenge) edge (attempt);
        \draw[thick, ->](attempt) -- node[center, fill=white]{sim}(success);
        \draw[thick, ->](attempt.south) |- (5,-2)
        node[center, above, xshift=0.25cm,fill=white]{não} -| (challenge.south);
        \draw[thick, ->](success) edge (end);
    \end{tikzpicture}
    \floatfoot{Adaptado de \cite{gilad2013}}
    \label{fig:diagrama-autenticacao-desafio-resposta}
\end{figure}

Segundo~\textcite{idrus2013} caso seja executada de forma adequada, a autenticação
via desafio-resposta criptográfica se mostra uma abordagem de autenticação extremamente
eficiente e confiável.
No entanto, a manutenção pode exigir um grande investimento de tempo, energia e
recursos financeiros.