\subsection{Autenticação Estática via Segredo Compartilhado}
\label{subsec:autenticacao-estatica-segredo-compartilhado}

A autenticação estática é amplamente utilizada na forma de autenticação baseada
em formulário de~\acrshort{id} e senha, sendo considerada a aplicação mais
conhecida.
Nesse método, o solicitante confirma sua identidade ao monitor, demonstrando
conhecimento de um segredo compartilhado, geralmente revelando-o para eles\cite{idrus2013}.

A autenticação estática geralmente funciona da
seguinte forma: o requerente verifica para o requerido que está ciente de um
segredo
compartilhado, geralmente divulgando-o para tal.
O requerido então determina se é o valor correto comparando-o com um mantido em
um banco de dados de verificação\cite{idrus2013}.

\begin{figure}[h!]
    \caption[Diagrama representação autenticação de \acrshort{id} e senha]
    {Diagrama de transição de estados representando autenticação com
    \acrshort{id} e senha}
    \begin{tikzpicture}
    [
        action/.style={draw, very thick, rounded corners, text width=2.5cm,
        align=center},
        initial/.style={draw, circle, fill, minimum size=10mm},
        final/.style={draw, double, thick, circle, fill, minimum size=10mm},
        decision/.style={draw, diamond, black, fill, minimum size=10mm},
    ]
        \node[initial](init);
        \node[action](provision)[right=0.5cm of init]{O usuário fornece nome de
        usuário e senha.};
        \node[decision, label={
            [text width=2.5cm, xshift=0.5cm]Nome de usuário e senha corretos?}]
        (validation)[right=0.5cm of provision];
        \node[action, text width=3cm](error)[below=1cm of validation]
        {Nome de usuário e/ou senha inválidos. Solicite ao usuário novamente.}
        \node[action](success)[right=1.5cm of validation]
        {Autenticação do usuário bem-sucedida. Garantir acesso.};
        \node[final](end)[right=0.5cm of success];
        
        \draw[thick, ->](init) edge (provision);
        \draw[thick, ->](provision) edge (validation);
        \draw[thick, ->](validation) --node[center, fill=white]{sim} (success);
        \draw[thick, ->](validation) --node[center, fill=white]{não} (error);
        \draw[thick, ->](error) -| (provision);
        \draw[thick, ->](success) edge (end);
    \end{tikzpicture}
    \floatfoot{Adaptado de \cite{horsh2018}}
    \label{fig:diagrama-autenticacao-id-senha}
\end{figure}

A autenticação de usuários baseada em senhas em sistemas computacionais tem
sido a base da segurança de computadores por muitos anos.
Para manter um segredo compartilhado entre um usuário e um sistema de
computador,
a ideia de um \acrshort{id} e senha é prática e econômica\cite{conklin2004}.

Do ponto de vista dos aplicativos, era lógico adotar a autenticação de usuário
baseada em senha quando havia menos aplicativos em comparação com o número de
usuários.
No entanto, essa suposição de que cada usuário possui um número limitado de
programas
não é mais válida devido ao crescimento da Internet e à tendência de computa
ção ubíqua.
Em muitos sistemas, os usuários têm várias contas e precisam memorizar diversos
IDs e senhas\cite{conklin2004}.